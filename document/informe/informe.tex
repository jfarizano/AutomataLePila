\documentclass[12pt,a4paper]{article}
\usepackage[utf8]{inputenc}
\usepackage[T1]{fontenc}
\usepackage[spanish, mexico]{babel}
\usepackage[spanish]{layout}
\usepackage[article]{ragged2e}
\usepackage[pdftex]{graphicx}
\usepackage{textcomp}
\usepackage{amsmath}
\usepackage{amssymb}
\usepackage{amsthm}
\usepackage{fancyvrb}
\usepackage{xcolor}
\usepackage{listings}
\usepackage[pdftex,colorlinks,linkcolor=blue]{hyperref}
\usepackage[cm]{fullpage}

\lstset{
  basicstyle=\ttfamily,
  mathescape
}

\begin{document}

%-------------------------------------------------------------------------------
%-------------------------------------------------------------------------------

\begin{titlepage}
\thispagestyle{empty}
\begin{center}
  \includegraphics[bb=0 0 100 100]{UNR-escudo-1.jpg}
  \\[0.5cm]
  \textsc{\LARGE Universidad Nacional de Rosario}\\[1.5cm]
  
  % Title
  {\huge \textbf{Autómata Le Pila} \\[0.4cm]}
  {\large Diseño e implementación de un DSL para autómatas de pila no deterministas} \\
  \noindent
  
  \vfill
  \vfill
  \vfill
  {\Large Autor: \par}
  {\Large Juan Ignacio Farizano\par}

  \vfill
  % Bottom of the page
  Trabajo Práctico Final \\
  Análisis de Lenguajes de Programación \\
  Departamento de Ciencias de la Computaci\'on\\
  Facultad de Ciencias Exactas, Ingenier\'ia y Agrimensura\\
  Rosario, Santa Fe, Argentina\\[0.4cm]
  {\large \today} 
\end{center}
\end{titlepage}

%-------------------------------------------------------------------------------
%-------------------------------------------------------------------------------

\tableofcontents

\newpage

\section{Motivación}
Este proyecto nace a partir de las clases de Lenguajes Formales y Computabilidad,
donde tuvimos nuestro primer acercamiento a estas máquinas reconocedoras de lenguaje.

La idea principal es proveer a les alumnes que estén cursando segundo año una
herramienta donde puedan testear los autómatas que construyan en la práctica,
pudiendo ver si ciertas palabras del lenguaje pedido son reconocidas como esperado,
ver el paso a paso de como la máquina elige la siguiente transición y como cambia
su configuración en el momento (utilizando el modo \emph{verbose}) además de luego
poder graficar estos autómatas de forma sencilla.

\section{Decisiones de diseño}
Durante el tiempo transcurrido entre la primera presentación de la idea para el
trabajo y su primera versión funcional el lenguaje pasó por muchos cambios.

Al principio se iba permitir cargar en memorias varios autómatas y modificarlos
en tiempo real desde la consola, pero esto se mostró muy complejo además de 
innecesario ya que en casos de uso reales se estaría probando solo un autómata
a la vez, por lo que se optó cargar uno solo a la vez y desde un archivo para
mantener la consola interactiva lo maś sencilla y fácil de utilizar posible.

Otras ideas eran por ejemplo tener diferentes órdenes de evaluación, ya sea
que en cada paso se elija una transición aleatoria entre todas las disponibles o
recorrerlas a todas para ver por cuantos caminos se puedan reconocer una palabra,
pero de nuevamente complejizaba el código, no eran características indispensables
y podían traer problemas en la ejecución como caer en loops infinitos con 
mayor facilidad.

En la versión final del programa se carga un autómata desde un archivo y a
través de una consola interactiva se permite cargar archivos nuevos, recargar
el último archivo utilizado, imprimir en consola el autómata, graficarlo el mismo
con salida a un archivo y reconocer palabras. También se cuenta con un modo verbose
que hace que en cada comando pedido por consola se imprima información extra
al respecto, por ejemplo en la evaluación se imprime cada paso y decisión tomada
o en el proceso de graficar imprime los parámetros utilizados.

\newpage

\section{Lenguaje}

Defino la sintaxis concreta del DSL a partir de la siguiente gramática, donde
el autómata se lee a partir de la regla de producción \textcolor{blue}{PDA} 
(\textcolor{blue}{P}ush\textcolor{blue}{D}own \textcolor{blue}{A}utomaton),
y a partir de ella se leen todas las partes que lo componen.

\subsection{Sintáxis concreta}

\begin{Verbatim}[commandchars=\\\{\}]
\textcolor{blue}{PDA} ::= InputAlph '=' \textcolor{blue}{ALPH}';'
        StackAlph '=' \textcolor{blue}{ALPH}';'
        States '=' \textcolor{blue}{STATES}';'
        AccStates '=' \textcolor{blue}{STATES}';'
        Transitions '=' \textcolor{blue}{TRANSITIONS}';'

\textcolor{blue}{ALPH} ::= '\{' \textcolor{blue}{SYMBOLS} '\}' | '\{' '\}'

\textcolor{blue}{SYMBOLS} ::= \textcolor{blue}{SYMBOL} | \textcolor{blue}{SYMBOL}',' \textcolor{blue}{SYMBOLS}

\textcolor{blue}{STATES} ::= '\{' \textcolor{blue}{STATES'} '\}' | '\{' '\}'

\textcolor{blue}{STATES'} ::= \textcolor{blue}{STATE} | \textcolor{blue}{STATE}',' \textcolor{blue}{STS}

\textcolor{blue}{TRANSITIONS} ::= '\{' \textcolor{blue}{TRANSITIONS'} '\}' | '\{' '\}'

\textcolor{blue}{TRANSITIONS'} ::=  \textcolor{blue}{TRANSITION} | \textcolor{blue}{TRANSITION}',' \textcolor{blue}{TRANSITIONS'}

\textcolor{blue}{TRANSITION} ::= '('\textcolor{blue}{STATE}',' \textcolor{blue}{SYMBOL}',' \textcolor{blue}{SYMBOL}',' \textcolor{blue}{SYMBOL}',' \textcolor{blue}{STATE}')'
\end{Verbatim}

\subsection*{Notas extras de la sintaxis}
La regla de producción \textcolor{blue}{SYMBOL} es análoga a leer chars individuales,
solo que no se permiten ciertos caracteres reservados, como los espacios, los caracteres
de control y los caracteres reservados '\textcolor{red}{=}', '\textcolor{red}{,}', '\textcolor{red}{;}',
'\textcolor{red}{(}', '\textcolor{red}{)}', '\textcolor{red}{\{}', '\textcolor{red}{\}}'.

La regla de producción \textcolor{blue}{STATE} es análoga a leer un string pero solo
con los caracteres permitidos en \textcolor{blue}{SYMBOL}.

\subsection{Notación}
Para definir un autómata, se lo debe escribir en un archivo con extensión \textbf{.pda}
donde dentro se lo define siguiendo la gramática dada anteriormente, como
se puede ver en el siguiente ejemplo:

\begin{lstlisting}
InputAlph = {x, y};
StackAlph = {x, y, #};
States = {s0, s1, s2, s3};
AccStates = {s0, s3};
Transitions = {(s0, $\lambda$, $\lambda$, #, s1), (s1, x, $\lambda$, x, s1), (s1, y, x, $\lambda$, s2),
(s2, y, x, $\lambda$, s2), (s2, $\lambda$, #, $\lambda$, s3)}
\end{lstlisting}


En este autómata podemos ver que en cada línea se define:
\begin{itemize}
  \item El alfabeto de entrada compuesto por x e y.
  \item El alfabeto de pila compuesto por x, y el símbolo \#.
  \item Los estados s0, s1, s2 y s3.
  \item Los estados de aceptación s0 y s3.
  \item Las transiciones.
\end{itemize}

Para el estado inicial se utiliza el primer estado definido.

Las transiciones son tuplas compuestas por estado actual, símbolo a leer de la entrada,
símbolo que se extrae de la pila, símbolo que se inserta a la pila y el estado siguiente. \\


\textbf{Nota:} El símbolo $\lambda$ no es necesario incluirlo en los lenguajes de entrada y pila,
ya que está incluido por defecto.

\newpage

\section{Resolución}
\subsection{Parseo}
Para el parser se utilizó la herramienta Happy, un generador de parsers para Haskell,
que definiendo una gramática en Yacc y un lexer en Haskell te genera un archivo Haskell
con el parser. Este código se puede encontrar en el archivo \textbf{src/Parse.y}

\subsection{Uso de las mónadas}
En el archivo \textbf{src/Monad.hs} se encuentra definida la mónada \emph{MonadPDA}
que se utiliza en todo el programa, esta está basada en la mónada del compilador
de la materia Compiladores y me permite realizar operaciones de entrada/salida,
manejo de errores y llevar un estado global con todos los datos necesarios
para la ejecución.

\subsection{Evaluación}
El evaluador de autómatas se encuentra en el archivo \textbf{src/Eval.hs},
allí se exporta la función evaluadora \emph{evalPDA} que recibe un autómata y una palabra 
a reconocer, esta a su vez hace uso de la función \emph{evalPDA'} a la que le pasa
los argumentos anteriores más la configuración inicial de la máquina que es
con el estado inicial y la pila vacía.
La evaluación si se atasca en algún paso realiza backtracking, lo que permite
evaluar autómatas de naturaleza no determinista.

\subsection{Otros archivos}
\begin{itemize}
  \item En el archivo \textbf{src/Lang.hs} se encuentran las estructuras de datos
  utilizadas para representar a un autómata de pila, sus partes y una configuración del mismo.
  \item En el archivo \textbf{src/Global.hs} se encuentra la estructura de datos
  utilizada para llevar el estado global del sistema y todos sus datos.
  \item En el archivo \textbf{src/Lib.hs} se encuentra funciones utilizadas en
  el resto del programa, en este caso solo se exporta una función utilizada para verificar
  que un autómata dado sea válido y cumpla con su estructura y condiciones requeridas.
  \item En el archivo \textbf{src/PPrint.hs} se encuentran las funciones utilizadas
  para imprimir en consola de forma más clara y fácil de leer las diferentes estructuras
  utilizadas, mensajes de errores, mensajes que solo se imprimen con modo verbose, etc.
  \item En el archivo \textbf{src/Graphic.hs} se encuentran las funciones utilizadas
  para graficar un autómata y exportarlo a un archivo de salida.
  \item El archivo \textbf{app/Main.hs} es el archivo principal del programa,
  en él se realiza el loop interactivo con la consola con todo la interpretación
  y evaluación de comandos que se tienen que realizar en ella.
\end{itemize}

\newpage

\section{Instalación y uso}
\subsection{Requisitos}
Para poder usar el programa es necesario tener instalada la plataforma de Haskell
junto a Stack. 

Para poder graficar autómatas es necesario tener instalado el 
paquete \emph{graphviz}, en caso contrario este comando nunca se ejecutará.

\subsection{Instalación}
\begin{enumerate}
  \item Clonar el repositorio con el comando:
    \begin{verbatim}
      git clone https://github.com/jfarizano/AutomataLePila.git
    \end{verbatim}
  \item Moverse a la carpeta del programa creada por el comando anterior y compilar con el comando:
    \begin{verbatim}
      stack build
    \end{verbatim}
\end{enumerate}
Con esto ya tenemos el programa instalado y está listo para su uso.

\subsection{Manual de uso}
Para ejecutar el programa utilizamos los comandos:
\begin{verbatim}
  stack run
  stack run -- [FLAGS] [FILE]
\end{verbatim}
Donde el primero ejecutará el programa sin cargar ningún autómata ni 
recibirá flags, en este caso se podrá usar la consola pero solo se permitirá
cargar archivos o usar la ayuda hasta que se use el comando de cargar archivos y
se lea un autómata válido. A partir de este punto se puede utilizar los comandos restantes.

En la segunda opción se le puede dar un archivo de entrada directamente al programa
para que este lo lea directamente sin tener que darlo en la consola interactiva, 
además de poder darle flags para cambiar parámetros de su ejecución y del comando de graficar.

Las acciones y opciones de estas flags se pueden ver con el comando
\begin{verbatim}
  stack run -- -h
\end{verbatim}

Dónde se pueden ver las siguientes opciones:

\begin{itemize}
  \item \textbf{--verbose} activa el modo verbose, haciendo que en consola se imprima más
  información relevante a la ejecución, por ejemplo en la evaluación de un autómata
  se imprimen todas las transiciones y configuraciones posibles por las que se puede
  seguir en cada paso.
  \item \textbf{--hs} o \textbf{--hsep} recibe un número como argumento, de esta forma se puede cambiar
  la distancia horizontal entre nodos en el gráfico resultante del comando de graficar.
  Se recomienda dar una distancia de 2 o mayor cuando el autómata contiene muchos nodos
  o uno donde tiene muchas transiciones que loopean en el mismo estado.
  \item \textbf{--vs} o \textbf{--vsep} análoga a la flag anterior pero de distancia vertical entre todos.
  \item \textbf{--dpi} recibe un número como argumento y esta será la densidad de pixeles
  de la resolución, no la resolución en sí.
  \item \textbf{-t}, \textbf{--tr} o \textbf{--transparentbg} si se activa esta flag cuando se grafique un
  autómata el gráfico resultante saldrá con fondo transparente.
  \item \textbf{--h}, \textbf{-?} o \textbf{--help} muestra la lista de opciones dadas aquí también.
\end{itemize}

Una vez dentro del loop de la consola interactiva y con un autómata válido cargado 
podemos realizar diferentes comandos, entre ellos imprimir una pantalla de ayuda,
activar o desactivar el modo verbose, cargar o recargar un archivo,
imprimir en consola el autómata cargado o graficar y exportar el mismo. La ayuda
de estos comandos se puede ver escribiendo el comando '\textbf{:h}' en consola.

Los comandos disponibles son los siguientes:

\begin{itemize}
  \item \textbf{:verbose} o \textbf{:v} activa el modo verbose si estaba desactivado y viceversa.
  \item \textbf{:print} o \textbf{:p} imprime en consola el autómata cargado actualmente.
  \item \textbf{:graphic} o \textbf{:g} recibe un archivo como argumento y exporta un gráfico del autómata
  cargado a ese archivo.
  \item \textbf{:load} o \textbf{:l} recibe un archivo con un autómata y lo carga en memoria.
  \item \textbf{:reload} o \textbf{:r} recarga el último archivo cargado.
  \item \textbf{:help}, \textbf{:h} o \textbf{:?} muestra la lista de comandos.
  \item \textbf{:quit} o \textbf{:q} sale del programa.
\end{itemize}

Si no hay un autómata cargado los únicos comandos que se podran usar serán los de
cargar un archivo, recargar el último archivo dado (en caso de que este haya sido inválido),
mostrar la pantalla de ayuda, activar y desactivar verbose, o salir del programa.

\newpage

\section{Bibliografía, material utilizado y software de terceros.}
\begin{itemize}
  \item \href{https://dcc.fceia.unr.edu.ar/es/lcc/r213}
             {Diapositivas de clase de la materia Lenguajes Formales y Computabilidad por Pablo Verdes.}
  \item \href{https://github.com/damianarielm/lcc/blob/master/2%20Ano/Lenguajes%20Formales%20y%20Computabilidad/Lenguajes%20Formales.pdf}
             {Apunte de Lenguajes Formales y Computabilidad por Damián Ariel Marotte}
  \item \href{https://github.com/compiladores-lcc/compiladores2021/blob/main/src/MonadFD4.hs}
             {Mónada basada en la mónada del compilador de la materia Compiladores dada por Mauro Jaskelioff.}
  \item \href{https://www.haskell.org/happy/}
             {Generador de parsers Happy.}
  \item \href{https://hackage.haskell.org/package/cmdargs}
             {Librería CmdArgs para parseo de argumentos en consola del sistema.}
  \item \href{https://hackage.haskell.org/package/haskeline}
             {Librería Haskeline para entrada en la consola interactiva del programa.}
  \item \href{https://hackage.haskell.org/package/prettyprinter}
             {Librería PrettyPrinter para imprimir en consola de forma clara.}
  \item \href{https://hackage.haskell.org/package/graphviz}
             {Librería GraphViz utilizada para poder usar el programa de gráficos de mismo nombre dentro de Haskell.}
  \item \href{https://hackage.haskell.org/package/fgl-5.7.0.3}
             {Librería FGL utilizada para una representación intermedia de los autómatas como grafos en la librería anterior.}
\end{itemize}

\end{document}